\documentclass[12pt,a4paper]{article} % use larger type; default would be 10pt

\usepackage{fullpage}  %To make the art22icle go to the full page
\usepackage[a4paper, hmargin=1.2in,vmargin=1in]{geometry} %To set the margins. hmargin is for the left and right margin and vmargin is for the top and bottom margin
\usepackage{setspace}  %To use double spacing
\usepackage{graphicx}  %To include graphics
\usepackage{subfig}    %To place figures side-by-side
\usepackage{booktabs}  %To make professional looking tables
\usepackage{threeparttablex}% For Notes below table
\usepackage{rotating}  %To make sideways tables
\usepackage{float}     %To float figures and tables
\usepackage{amsmath} %Helps with typesetting math
\usepackage{verbatim} %To allow for verbatim text and multiline comments
\usepackage{caption}  %For a top and bottom caption or multiple captions
\usepackage[toc,page]{appendix} %To offset appendices
\usepackage{changepage} %To change the margins of the page on the fly,
\usepackage{tabularx}  
\usepackage[parfill]{parskip}
\usepackage[longnamesfirst]{natbib}
\usepackage[pagebackref=true,hyperfootnotes=false]{hyperref}
\hypersetup{
	colorlinks=true,
	linkcolor=blue,
	citecolor=blue,
}
\usepackage{afterpage}
%%%%%%%%%%%%%%%%%%%%%%%%%%%%%%%%%%%%%%%%%%%%%%%%%%%%%%%%%%%%%%%%%%%%%%%%%%%%%%%%
% Todo notes
\usepackage{lipsum}
\usepackage{xargs}                      % Use more than one optional parameter in a new commands
\usepackage[pdftex,dvipsnames]{xcolor}  % Coloured text etc.
% 
\usepackage[colorinlistoftodos,prependcaption,textsize=tiny]{todonotes}
\newcommandx{\unsure}[2][1=]{\todo[linecolor=red,backgroundcolor=red!25,bordercolor=red,#1]{#2}}
\newcommandx{\change}[2][1=]{\todo[linecolor=blue,backgroundcolor=blue!25,bordercolor=blue,#1]{#2}}
\newcommandx{\info}[2][1=]{\todo[linecolor=OliveGreen,backgroundcolor=OliveGreen!25,bordercolor=OliveGreen,#1]{#2}}
\newcommandx{\improvement}[2][1=]{\todo[linecolor=Plum,backgroundcolor=Plum!25,bordercolor=Plum,#1]{#2}}
\newcommandx{\thiswillnotshow}[2][1=]{\todo[disable,#1]{#2}}
%%%%%%%%%%%%%%%%%%%%%%%%%%%%%%%%%%%%%%%%%%%%%%%%%%%%%%%%%%%%%%%%%%%%%%%%%%%%%%%%


%%%%%%%%%%%%%%%%%%%%%%%%%%%%%%%%%%%%%%%%%%%%%%%%%%%%%%%%%%%%%%%%%%%%%%%%%%%%%%%%
% TO MAKE ESTOUT WORK BETTER http://tex.stackexchange.com/questions/97764/regression-tables-made-with-esttab-in-stata-have-columns-with-different-widths

% *****************************************************************
% siunitx
% *****************************************************************
\newcommand{\sym}[1]{\rlap{#1}} % Thanks to Joseph Wright & David Carlisle

\usepackage{siunitx}
\sisetup{
	detect-mode,
	group-digits		= true,
	input-symbols		= ( ) [ ] - +,
	table-align-text-post	= false,
	input-signs             = ,
	group-separator = {,}
}

% Character substitution that prints brackets and the minus symbol in text mode. Thanks to David Carlisle
\def\yyy{%
	\bgroup\uccode`\~\expandafter`\string-%
	\uppercase{\egroup\edef~{\noexpand\text{\llap{\textendash}\relax}}}%
	\mathcode\expandafter`\string-"8000 }

\def\xxxl#1{%
	\bgroup\uccode`\~\expandafter`\string#1%
	\uppercase{\egroup\edef~{\noexpand\text{\noexpand\llap{\string#1}}}}%
	\mathcode\expandafter`\string#1"8000 }

\def\xxxr#1{%
	\bgroup\uccode`\~\expandafter`\string#1%
	\uppercase{\egroup\edef~{\noexpand\text{\noexpand\rlap{\string#1}}}}%
	\mathcode\expandafter`\string#1"8000 }

\def\textsymbols{\xxxl[\xxxr]\xxxl(\xxxr)\yyy}


\makeatletter
\newcommand*{\centerfloat}{%
	\parindent \z@
	\leftskip \z@ \@plus 1fil \@minus \textwidth
	\rightskip\leftskip
	\parfillskip \z@skip}
\makeatother


% *****************************************************************
% Estout related things
% *****************************************************************
\let\estinput=\input % define a new input command so that we can still flatten the document

\newcommand{\estwide}[3]{
	\vspace{.75ex}{
		\textsymbols% Note the added command here
		\begin{tabular*}
			{\textwidth}{@{\hskip\tabcolsep\extracolsep\fill}l*{#2}{#3}}
			\toprule
			\estinput{#1}
			\bottomrule
			\addlinespace[.75ex]
		\end{tabular*}
	}
}

\newcommand{\estauto}[3]{
	\vspace{.75ex}{
		\textsymbols% Note the added command here
		\begin{tabular}{l*{#2}{#3}}
			\toprule
			\toprule
			\estinput{#1}
			\bottomrule
			\bottomrule
			\addlinespace[.75ex]
		\end{tabular}
	}
}

% Allow line breaks with \\ in specialcells
\newcommand{\specialcell}[2][c]{%
	\begin{tabular}[#1]{@{}c@{}}#2\end{tabular}
}

%%%%%%%%%%%%%%%%%%%%%%%%%%%%%%%%%%%%%%%%%%%%%%%%%%%%%%%%%%%%%%%%%%%%%%%%%%%%%%%%
% *****************************************************************
% Custom subcaptions
% *****************************************************************
% Note/Source/Text after Tables
% The new approach using threeparttables to generate notes that are the exact width of the table.
\newcommand{\Figtext}[1]{%
	\begin{tablenotes}[para,flushleft]
		\hspace{6pt}
		\hangindent=1.75em
		#1
	\end{tablenotes}
}
\newcommand{\Fignote}[1]{\Figtext{\emph{Note:~}~#1}}
\newcommand{\Figsource}[1]{\Figtext{\emph{Source:~}~#1}}
\newcommand{\Starnote}{\Figtext{* p < 0.1, ** p < 0.05, *** p < 0.01. Standard errors in parentheses.}}% Add significance note with \starnote
%%%%%%%%%%%%%%%%%%%%%%%%%%%%%%%%%%%%%%%%%%%%%%%%%%%%%%%%%%%%%%%%%%%%%%%%%%%%%%%%


%%%%%%%%%%%%%%%%%%%%%%%%%%%%%%%%%%%%%%%%%%%%%%%%%%%%%%%%%%%%%%%%%%%%%%%%%%%%%%%%
% Citation manager 
%%%%%%%%%%%%%%%%%%%%%%%%%%%%%%%%%%%%%%%%%%%%%%%%%%%%%%%%%%%%%%%%%%%%%%%%%%%%%%%%
%\usepackage[style=chicago-authordate]{biblatex}

\doublespacing
\begin{document}


	\author{
		Claes B\"{a}ckman
	}
			
	\title{Ska du köpa eller hyra? En guide till Verktyget}
	\date{\today}
	
	\maketitle
	
%	
%\begin{abstract}
%\begin{singlespace}
%
%\end{singlespace}
%\end{abstract}


	Den här guiden beskriver hur verktyget räknar fram saker. 

\section{Beräkning av motsvarande hyreskostnad}
Hyreskostnader består av initiala kostnader, löpande kostnader, alternativkostnader och vinster, precis som för att köpa. 

\subsubsection*{Initiala kostnader}
Initiala kostnader för att hyra är deposition och eventuella avgifter för att hitta en lägenhet. 

\begin{equation}
	Initiala_h = h_1 * d
\end{equation}
där $h$ är hyra första året och $d$ är depositionstid. Så initiala kostnader räknas som deposition gånger den initiala hyran. 

\subsubsection*{Löpande kostnader}
Löpande kostnader består av hyreskostnader och summan av andra kostnader: 
\begin{equation}
	LK_h = h_1* \frac{1-(1+x)^t}{1-(1+x)} + AndraHyra *  t
\end{equation}
där $x$ är ökning i hyran och $AndraHyra$ är årskostnaden för andra kostnader för att hyra. 


\subsubsection*{Alternativkostnad}
Alternativkostnad med ökande hyra varje år beräknas enligt:\footnote{Se \url{https://money.stackexchange.com/questions/94899/which-compound-interest-formula-can-i-use-to-find-the-final-balance-with-monthly}.}
\begin{equation}
	fv = \frac{(p (1 + r) (-1 + (1 + r)^m) ((1 + r)^{(m y)} - (1 + x)^y))
}{	(r (-1 + (1 + r)^m - x))}
\end{equation}

Där r är månatliga eller kvartsmässiga räntan, y är antal år, m är antal perioder per år, p är den ursprungliga hyran eller insatsen, och x är ökningen i den unsprungliga hyran. Om $m=$ kan vi skriva formeln som:
\begin{equation}
		fv = \frac{(p (1 + r)) ((1 + r)^y - (1 + x)^y))
	}{	(r - x)}
\end{equation}
För att beräkna alternativkostnaden får vi lägga till alternativkostnaden för depositionen också, samt dra bort summan av hyran och depositionen. 
\begin{eqnarray}
	AK_h &= fv - \sum_{i=1}^{10} h_i + hd * (1+r)^t - hd  \\
		  &=\frac{(h_1 (1 + r)) ((1 + r)^y - (1 + x)^y))}{	(r - x)} - h_1* \frac{1-(1+x)^t}{1-(1+x)} +  hd * (1+r)^t - hd
\end{eqnarray}


\subsubsection*{Vinster}
Vinster är helt enkelt en återbetald deposition. 
\begin{equation}
	Vinster_h = -1* Initiala_h 
\end{equation}

\subsubsection*{Sammanlagda kostnader}
De sammanlagda kostnaderna är helt enkelt summan av de ovanstående kostnaderna: 
\begin{eqnarray*}
	Sammanlagt &= Initiala_h + LK_h + AK_h + Vinster_h \\
	&= h_1 d + h_1* \frac{1-(1+x)^t}{1-(1+x)} + AndraHyra *  t  \\ 
	&+ \frac{(h_1 (1 + r) ((1 + r)^y - (1 + x)^y))}{(r - x)} - \\ 
	&h_1* \frac{1-(1+x)^t}{1-(1+x)} +  hd * (1+r)^t - h_1 d  -h_1 d 
\end{eqnarray*}
Efter att ha skrivit om har vi:
\begin{eqnarray*}
		&Sammanlagt = h_1 d- h_1 d  -h_1 d  +  h_1* \frac{1-(1+x)^t}{1-(1+x)} - h_1* \frac{1-(1+x)^t}{1-(1+x)}  \\
		 &+ AndraHyra *  t + \frac{(h_1 (1 + r) ((1 + r)^y - (1 + x)^y))}{(r - x)} +  hd * (1+r)^t \\
	&=  AndraHyra *  t + \frac{(h_1 (1 + r) ((1 + r)^y - (1 + x)^y))}{(r - x)} +  hd * (1+r)^t  - h_1 d 
%	& =  AndraHyra *  t  + \frac{(h_1 (1 + r)) ((1 + r)^y - (1 + x)^y))}{(r - x)} +  hd * (1+r)^t  - h_1 d
\end{eqnarray*}

Nu vill vi att sammanlagda kostnader ska bli lika med kostnaden för att köpa:
\begin{eqnarray*}
	&Sammanlagda_k = Sammanlagda_h  \\
	& sammanlagda_k =  AndraHyra *  t + \frac{(h_1 (1 + r) ((1 + r)^y - (1 + x)^y))}{(r - x)} +  h_1d * (1+r)^t  - h_1 d \\
	& sammanlagda_k -  AndraHyra *  t = \frac{h_1 ((1 + r) ((1 + r)^y - (1 + x)^y)) }{(r - x)} +  h_1d * (1+r)^t  - h_1 d \\
\end{eqnarray*}
Om vi bryter ut $h_1$: 
\begin{eqnarray*}
		& sammanlagda_k -  AndraHyra *  t = \\
		&h_1 ((1 + r) ((1 + r)^y - (1 + x)^y)+ (r-x)d(1+r)^t -d(r-x)) \frac{1}{(r - x)} \\
\end{eqnarray*}
Så kan vi skriva om så att vi får en hyra baserat på totala kostnader för att köpa och hyra: 
\begin{eqnarray*}
	& \frac{(sammanlagda_k -  AndraHyra *  t)(r - x)}{((1 + r) ((1 + r)^y - (1 + x)^y)+ (r-x)d(1+r)^t -d(r-x))} = h_1  \\
\end{eqnarray*}

%
%Vinster: 
%Vinster=-1*deposition
%
%
%Hur vi hittar hyreskostnaden per månad: 
%kostnad_köpa=initiala+ löpande+alternativkostnader-vinster
%=deposition+ Total hyra+andra kostnader för att hyra+fv-Total hyra+deposition*(1+r)^tid-deposition
%hd+h*((1-(1+öknin〖g)〗^tid)/(1-(1+ökning) ))+AndraHyra*12*tid+  (hyra_1*(1+r)*[(1+r)^tid-(1+hyresökning)^tid])/((r-hyresökning))
%
%







%	\bibliographystyle{authordate1}
%	\bibliography{my_library}
	

%		\input{Figures.tex}

	
%	\listoftodos[Notes]
	
\end{document}
