Den här guiden beskriver hur verktyget räknar fram saker. 

\section{Beräkning av motsvarande hyreskostnad}
Hyreskostnader består av initiala kostnader, löpande kostnader, alternativkostnader och vinster, precis som för att köpa. 

\subsubsection*{Initiala kostnader}
Initiala kostnader för att hyra är deposition och eventuella avgifter för att hitta en lägenhet. 

\begin{equation}
	Initiala_h = h_1 * d
\end{equation}
där $h$ är hyra första året och $d$ är depositionstid. Så initiala kostnader räknas som deposition gånger den initiala hyran. 

\subsubsection*{Löpande kostnader}
Löpande kostnader består av hyreskostnader och summan av andra kostnader: 
\begin{equation}
	LK_h = h_1* \frac{1-(1+x)^t}{1-(1+x)} + AndraHyra *  t
\end{equation}
där $x$ är ökning i hyran och $AndraHyra$ är årskostnaden för andra kostnader för att hyra. 


\subsubsection*{Alternativkostnad}
Alternativkostnad med ökande hyra varje år beräknas enligt:\footnote{Se \url{https://money.stackexchange.com/questions/94899/which-compound-interest-formula-can-i-use-to-find-the-final-balance-with-monthly}.}
\begin{equation}
	fv = \frac{(p (1 + r) (-1 + (1 + r)^m) ((1 + r)^{(m y)} - (1 + x)^y))
}{	(r (-1 + (1 + r)^m - x))}
\end{equation}

Där r är månatliga eller kvartsmässiga räntan, y är antal år, m är antal perioder per år, p är den ursprungliga hyran eller insatsen, och x är ökningen i den unsprungliga hyran. Om $m=$ kan vi skriva formeln som:
\begin{equation}
		fv = \frac{(p (1 + r)) ((1 + r)^y - (1 + x)^y))
	}{	(r - x)}
\end{equation}
För att beräkna alternativkostnaden får vi lägga till alternativkostnaden för depositionen också, samt dra bort summan av hyran och depositionen. 
\begin{eqnarray}
	AK_h &= fv - \sum_{i=1}^{10} h_i + hd * (1+r)^t - hd  \\
		  &=\frac{(h_1 (1 + r)) ((1 + r)^y - (1 + x)^y))}{	(r - x)} - h_1* \frac{1-(1+x)^t}{1-(1+x)} +  hd * (1+r)^t - hd
\end{eqnarray}


\subsubsection*{Vinster}
Vinster är helt enkelt en återbetald deposition. 
\begin{equation}
	Vinster_h = -1* Initiala_h 
\end{equation}

\subsubsection*{Sammanlagda kostnader}
De sammanlagda kostnaderna är helt enkelt summan av de ovanstående kostnaderna: 
\begin{eqnarray*}
	Sammanlagt &= Initiala_h + LK_h + AK_h + Vinster_h \\
	&= h_1 d + h_1* \frac{1-(1+x)^t}{1-(1+x)} + AndraHyra *  t  \\ 
	&+ \frac{(h_1 (1 + r) ((1 + r)^y - (1 + x)^y))}{(r - x)} - \\ 
	&h_1* \frac{1-(1+x)^t}{1-(1+x)} +  hd * (1+r)^t - h_1 d  -h_1 d 
\end{eqnarray*}
Efter att ha skrivit om har vi:
\begin{eqnarray*}
		&Sammanlagt = h_1 d- h_1 d  -h_1 d  +  h_1* \frac{1-(1+x)^t}{1-(1+x)} - h_1* \frac{1-(1+x)^t}{1-(1+x)}  \\
		 &+ AndraHyra *  t + \frac{(h_1 (1 + r) ((1 + r)^y - (1 + x)^y))}{(r - x)} +  hd * (1+r)^t \\
	&=  AndraHyra *  t + \frac{(h_1 (1 + r) ((1 + r)^y - (1 + x)^y))}{(r - x)} +  hd * (1+r)^t  - h_1 d 
%	& =  AndraHyra *  t  + \frac{(h_1 (1 + r)) ((1 + r)^y - (1 + x)^y))}{(r - x)} +  hd * (1+r)^t  - h_1 d
\end{eqnarray*}

Nu vill vi att sammanlagda kostnader ska bli lika med kostnaden för att köpa:
\begin{eqnarray*}
	&Sammanlagda_k = Sammanlagda_h  \\
	& sammanlagda_k =  AndraHyra *  t + \frac{(h_1 (1 + r) ((1 + r)^y - (1 + x)^y))}{(r - x)} +  h_1d * (1+r)^t  - h_1 d \\
	& sammanlagda_k -  AndraHyra *  t = \frac{h_1 ((1 + r) ((1 + r)^y - (1 + x)^y)) }{(r - x)} +  h_1d * (1+r)^t  - h_1 d \\
\end{eqnarray*}
Om vi bryter ut $h_1$: 
\begin{eqnarray*}
		& sammanlagda_k -  AndraHyra *  t = \\
		&h_1 ((1 + r) ((1 + r)^y - (1 + x)^y)+ (r-x)d(1+r)^t -d(r-x)) \frac{1}{(r - x)} \\
\end{eqnarray*}
Så kan vi skriva om så att vi får en hyra baserat på totala kostnader för att köpa och hyra: 
\begin{eqnarray*}
	& \frac{(sammanlagda_k -  AndraHyra *  t)(r - x)}{((1 + r) ((1 + r)^y - (1 + x)^y)+ (r-x)d(1+r)^t -d(r-x))} = h_1  \\
\end{eqnarray*}

%
%Vinster: 
%Vinster=-1*deposition
%
%
%Hur vi hittar hyreskostnaden per månad: 
%kostnad_köpa=initiala+ löpande+alternativkostnader-vinster
%=deposition+ Total hyra+andra kostnader för att hyra+fv-Total hyra+deposition*(1+r)^tid-deposition
%hd+h*((1-(1+öknin〖g)〗^tid)/(1-(1+ökning) ))+AndraHyra*12*tid+  (hyra_1*(1+r)*[(1+r)^tid-(1+hyresökning)^tid])/((r-hyresökning))
%
%





